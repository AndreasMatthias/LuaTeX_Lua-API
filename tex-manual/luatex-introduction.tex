% language=uk

\environment luatex-style

\startcomponent luatex-introduction

\startchapter[title=Introduction]

This is the reference manual of \LUATEX. We don't claim it is complete and we
assume that the reader knows about \TEX\ as described in \quotation {The \TEX\
Book}, the \quotation {\ETEX\ manual}, the \quotation {\PDFTEX\ manual}, etc.
Additional reference material is published in journals of user groups and
\CONTEXT\ related documentation.

It took about a decade to reach stable version 1.0, but for good reason.
Successive versions brought new functionality, more control, some cleanup of
internals. Experimental features evolved into stable ones or were dropped.
Already quite early \LUATEX\ could be used for production and it was used on a
daily basis by the authors. Successive versions sometimes demanded an adaption to
the \LUA\ interfacing, but the concepts were unchanged. The current version can
be considered stable in functionality and there will be no fundamental changes.
Of course we then can decide to move towards version 2.00 with different
properties.

Don't expect \LUATEX\ to behave the same as \PDFTEX ! Although the core
functionality of that 8 bit engine was starting point, it has been combined with
the directional support of \OMEGA\ (\ALEPH). But, \LUATEX\ can behave different
due to its wide (32 bit) characters, many registers and large memory support. The
\PDF\ code produced differs from \PDFTEX\ but users will normally not notice
that. There is native \UTF\ input, support for large (more than 8 bit) fonts, and
the math machinery is tuned for \OPENTYPE\ math. There is support for directional
typesetting too. The log output can differ from other engines and will likely
differ more as we move forward. When you run plain \TEX\ for sure \LUATEX\ runs
slower than \PDFTEX\ but when you run for instance \CONTEXT\ \MKIV\ in many cases
it runs faster, especially when you have a bit more complex documents or input.
Anyway, 32 bit all||over combined with more features has a price, but on a modern
machine this is no real problem.

Testing is done with \CONTEXT, but \LUATEX\ should work fine with other macro
packages too. For that purpose we provide generic font handlers that are mostly
the same as used in \CONTEXT. Discussing specific implementations is beyond this
manual. Even when we keep \LUATEX\ lean and mean, we already have enough to
discuss here.

\LUATEX\ consists of a number of interrelated but (still) distinguishable parts.
The organization of the source code is adapted so that it can glue all these
components together. We continue cleaning up side effects of the accumulated
code in \TEX\ engines (especially code that is not needed any longer).

\startitemize [unpacked]
    \startitem
        We started out with most of \PDFTEX\ version 1.40.9. The code base was
        converted to \CCODE\ and split in modules. Experimental features were
        removed and utility macros are not inherited because their functionality
        can be programmed in \LUA. The number of backend interface commands has
        been reduced to a few. The so called extensions are separated from the
        core (which we try to keep close to the original \TEX\ core). Some
        mechanisms like expansion and protrusion can behave different from the
        original due to some cleanup and optimization. Some whatsit based
        functionality (image support and reusable content) is now core
        functionality. We don't stay in sync with \PDFTEX\ development.
    \stopitem
    \startitem
        The direction model from \ALEPH\ RC4 (which is derived from \OMEGA) is
        included. The related primitives are part of core \LUATEX\ but at the
        node level directional support is no longer based on so called whatsits
        but on real nodes with relevant properties. The number of directions is
        limited to the useful set and the backend has been made direction aware.
    \stopitem
    \startitem
        Neither \ALEPH's I/O translation processes, nor tcx files, nor \ENCTEX\
        are available. These encoding|-|related functions are superseded by a
        \LUA|-|based solution (reader callbacks). In a similar fashion all file
        \IO\ can be intercepted.
    \stopitem
    \startitem
        We currently use \LUA\ 5.3.*. There are few \LUA\ libraries that we
        consider part of the core \LUA\ machinery, for instance \type {lpeg}.
        There are additional \LUA\ libraries that interface to the internals of
        \TEX. We also keep the \LUA\ 5.2 \type {bit32} library around.
    \stopitem
    \startitem
        There are various \TEX\ extensions but only those that cannot be done
        using the \LUA\ interfaces. The math machinery often has two code paths:
        one traditional and the other more suitable for wide \OPENTYPE\ fonts. Here
        we follow the \MICROSOFT\ specifications as much as possible.  Some math
        functionality has been opened up a bit so that users have more control.
    \stopitem
    \startitem
        The fontloader uses parts of \FONTFORGE\ 2008.11.17 combined with
        additional code specific for usage in a \TEX\ engine. We try to minimize
        specific font support to what \TEX\ needs: character references and
        dimensions and delegate everything else to \LUA. That way we keep \TEX\
        open for extensions without touching the core. In order to minimize
        dependencies at some point we may decide to make this an optional
        library.
    \stopitem
    \startitem
        The \METAPOST\ library is integral part of \LUATEX. This gives \TEX\ some
        graphical capabilities using a relative high speed graphical subsystem.
        Again \LUA\ is used as glue between the frontend and backend. Further
        development of \METAPOST\ is closely related to \LUATEX.
    \stopitem
    \startitem
        The virtual font technology that comes with \TEX\ has been integrated
        into the font machinery in a way that permits creating virtual fonts
        at runtime. Because \LUATEX\ can also act as a \LUA\ interpreter this
        means that a complete \TEX\ workflow can be built without the need for
        additional programs.
    \stopitem
    \startitem
        The versions starting from 1.09 no longer use the poppler library for
        inclusion but a lightweight dedicated one. This removes a dependency but
        also makes the inclusion code of \LUATEX\ different from \PDFTEX. In fact
        it was already much different due to the \LUA\ image interfacing.
    \stopitem
\stopitemize

We try to keep upcoming versions compatible but intermediate releases can contain
experimental features. A general rule is that versions that end up on \TEXLIVE\
and|/|or are released around \CONTEXT\ meetings are stable. Any version between
the yearly \TEXLIVE\ releases are to be considered beta and in the repository end
up as trunk releases. We have an experimental branch that we use for development
but there is no support for any of its experimental features. Intermediate
releases (from trunk) are normally available via the \CONTEXT\ distribution
channels (the garden and so called minimals).

Version 1.10 is more or less an endpoint in development: this is what you get.
Because not only \CONTEXT, that we can adapt rather easily, uses \LUATEX, we
cannot change fundamentals without unforeseen consequences. By now it has been
proven that \LUA\ can be used to extend the core functionality so there is no
need to add more, and definitely no hard coded solutions for (not so) common
problems. Of course there will be bug fixes, maybe some optimization, and there
might even be some additions or non|-|intrusive improvements, but only after
testing outside the stable release. After all, the binary is already more than
large enough and there is not that much to gain.

You might find \LUA\ helpers that are not yet documented. These are considered
experimental, although when you encounter them in a \CONTEXT\ version that has
been around for a while you can assume that they will stay. Of course it can just
be that we forgot to document them yet.

A manual like this is not really meant as tutorial, for that we refer to
documents that ship with \CONTEXT, articles, etc. It is also never complete
enough for all readers. We try to keep up but the reader needs to realize that
it's all volunteer work done in spare time. And for sure, complaining about a bad
manual or crappy documentation will not really motivate us to spend more time on
it. That being said, we hope that this document is useful.

\blank[big]

\testpage[5]

\startlines
Hans Hagen
Harmut Henkel
Taco Hoekwater
Luigi Scarso
\stoplines

\blank[3*big]

\starttabulate[|||]
\NC Version  \EQ \currentdate \NC \NR
\NC \LUATEX  \EQ \cldcontext{LUATEXENGINE} %
                 \cldcontext{"\letterpercent 0.2f",LUATEXVERSION} / %
                 \cldcontext{LUATEXFUNCTIONALITY}
                 \NC \NR
\NC \CONTEXT \EQ MkIV \contextversion \NC \NR
\stoptabulate

\stopchapter

\stopcomponent
